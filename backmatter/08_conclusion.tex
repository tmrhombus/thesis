
\chapter{Conclusions and Future Prospects}\label{sec:conclusion}

In this thesis, two analyses of data collected by the CMS
 collaboration using $pp$ collisions provided by the LHC are presented.

The SM process \ppwbblnbb is studied at \s8 \TeV
 using a data sample that corresponds to an
 integrated luminosity of {19.8~fb$^{-1}$}.
The $W$ boson is identified by an isolated lepton ($\mu$ or $e$)
 with $\pt^\ell>30$ \GeV and $\abs{\eta^\ell}<2.1$.
Backgrounds from \ppttbar and Drell--Yan processes
 are reduced by rejecting events with a second lepton 
 within {$\pt > 10$} GeV and {$|\eta| < 2.4$}.
Exactly two $b$-tagged jets with
 {$\pt > 25$} GeV and {$|\eta| < 2.4$} are required
 to be present in selected events, to  
 remove contamination in the signal region from 
 charm and light flavor jets.
To reduce the contribution from \ppttbar events,
 events  with
 a third jet with {$\pt > 25$} GeV and {$|\eta| < 4.7$}
 are rejected.

Fits are performed in sidebands dominated by \ttbar
 events  to adjust the  simulated jet energy scale
 as well as the
 scale factor associated with the difference in
 efficiency between data and simulation for the identification of $b$ quarks.
After making these adjustments, a fit is performed in the
 signal region and the cross section is extracted as
{$\sigma ( {\mathrm{pp}} \rightarrow {\mathrm{W}} (\ell\nu)$+$\mathrm{b}\overline{\mathrm{b}})= 0.64 \pm 0.03 \mathrm{(stat)} \pm 0.10 \mathrm{(syst)} \pm 0.06 \mathrm{(theo)} \pm 0.02 \mathrm{(lumi)} ~\mathrm{pb}$}.
This cross section is compared with four SM predictions made using
 \MCFM and \MADGRAPH+\PYTHIA with varied PDFs and is
 found to be compatible.

The other analysis is of the monophoton signature
 and is performed using data corresponding to 2.3~\fbinv
 at \s13 \TeV.
The data are selected requiring one isolated photon
 with $\ptg>175$~\GeV and $\abs{\eta} < 1.44$ and
 events are vetoed if they contain 
 a charged lepton (an electron or a muon) with $\pt >10$~\GeV
 that is separated from the photon by $\Delta R > 0.5$ radians.
The monophoton signature is one where the 
 photon recoils from the interaction with some
 particle(s) that do not leave a trace in the detector
 so events are required to have $\met > 170$~\GeV.
To ensure that the main source of \met is
 not photon energy mismeasurement, 
 the azimuthal opening
 angle between the candidate photon and $\vmet$ is required to be
 greater than 2 radians.
Jet energy mismeasurement can also
 give rise to \met so, events are rejected if the minimum azimuthal opening
 angle between $\vmet$ and up to four leading jets (\minDphiMETj) is
 less than 0.5 radians.

Interpreting these results as a measurement of the SM
 cross section for invisible decays of the $Z$ boson
 the cross section is measured as
 $\sigma(\ppzgnng) 64.06 \pm 12.14(\mathrm{stat}) \pm 12.88(\mathrm{syst}) \pm 1.72(\mathrm{lumi})\;\; \fb $
 which is in agreement with the theoretical value calculated at NNLO
 of $65.55\pm 0.02$ \fbinv.
These results are also interpreted in the context of a search for DM
 using simplified models with a vector or axial-vector mediator and as an
 EFT coupling vertex
 $\gamma\gamma\chi\overline{\chi}$.
 which allows for DM production via 
 the channel
 $pp\rightarrow\gamma\rightarrow\gamma\chi\overline{\chi}$.
No evidence for DM has been found, and limits on the
 parameters in these models are set.

In the simplified model assuming a DM mass $m_\chi < 10$ \GeV, the mediator
 mass is found to be $m_M \nless 600$ \GeV assuming either
 vector or axial-vector couplings.
In the EFT model, lower limits are on the
 coupling strength suppression scale 
 are presented as a function of $m_\chi$ and are 
 $\Lambda < 540$ \GeV is excluded at $95\%$ CL.

The LHC continues to provide $pp$ collisions at \s13 \TeV
 which are presently being collected and analyzed by the CMS collaboration.
Only 2.3~\fbinv were analyzed in the monophoton 
 analysis presented in this thesis, and 
 data continues be collected.
The statistical uncertainty presented in this monophoton analyses
 is comparable with the systematic uncertainty
 and will decrease with more data as the number of expected
 events scales linearly with the integrated luminosity.
With $30-40$~\fbinv of data projected to be
 collected by the end of 2016, this corresponds to 
 an expected 1000-1300 events clearly identified
 in the monophoton final state.

With the discovery of the Higgs boson
 at the LHC, all fundamental particles
 predicted by the SM
 have now been observed.
Searches for physics beyond the SM
 and for DM in particular are therefore 
 an exciting field of study
 and could possibly lead to new understandings of
 the material hypothesized to constitute the majority 
 of mass throughout the universe.
 



%Backgrounds from the halo of particles travelling
% roughly collinear with the beams are reduced
% by rejecting events with MIP > 4.9 GeV

