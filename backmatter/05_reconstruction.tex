
\chapter{Event Reconstruction}\label{sec:reconstruction}

 Real data collected from the detector
  and simulated data output from Geant4
  consist of time-correlated energy deposits
  in the various subdetectors of CMS.
 As a result of the coordinated designs 
  of the subdetectors, the final-state 
  particles which arise from $pp$ collisions 
  at the LHC can be individually identified
  and reconstructed using the combined
  information from the entirity of CMS.
 The associated global event description
  from this particle-flow (PF) reconstruction
  provides excellent performance for
  the identification of electrons and muons,
  as well as for vertex identification
  and the evaluation of \met.
 
\section{Track and Primary Vertex Reconstruction}
 The subdetector closest to the interaction vertex
  is the tracker, which records precise
  information about the trajectories of 
  charged particles as they pass through it.
 Combined with the magnetic field, 
  this allows for the measurement of the
  momenta of these particles as well as a
  means of identifying the the location of
  the primary interaction.

 Tracks are identified via an iteritive process. 
 The first tracks to be reconstructed
  are those which pass strict seeding
  criteria, designed to have a moderate
  efficiency, but negligibly small
  fake rate.
 Then the detector hits associated
  with these tracks are masked
  and the remaining hits are used to
  form track seeds with slightly relaxed
  criteria.
 This operation is repeated, with every
  iteration imposing more complex and time-consuming
  seeding, filtering and track fitting algorithms.
 
% In 2012, the average number of interactions
%  happening at every bunch crossing was 21,
%  and in 2015 it was 
 Because bunches of protons instead of single protons
  are made to cross in the LHC, 
  multiple collisions can take place during the same
  bunch crossing.
 The vertex with the highest scalar sum
  transverse momentum, \pt,
  of tracks and passing further quality selections
  based on the goodness of fit for the tracks
  and the number of tracks associated with a given vertex
  is chosen as the primary vertex (PV).

 In \ppwbblnbb events, the two $b$ quarks and
  the lepton from the $W$ decay all leave energy
  deposits in the tracker, thus making the choice
  of PV unambiguous.
 However, in the \pploneg events,
  the only visible final state object is a photon,
  and photons do not leave hits in the tracker.
 This makes the identification of the PV
  in the monophoton analysis difficult
  and motivates the using of variables that
  are less sensitive to correct PV identification.
  
\section{Electron ID and Reconstruction}

 Electrons are reconstructed using tracker
  hits and ECAL deposits.
 The seed of an electron candidate is selected as 
  an energy deposit in the ECAL with $E_T > 4$ \GeV
  having nearby deposits in the tracker.
 As electrons move in magnetic fields,
  they emit bremsstrahlung radiation
  tangental to their flight path
  and this radiation both appears in the detector, 
  and alters the course of the electron.

 The effects of this radiation are taken
  into account via the Gaussian Sum Filter (GSF)
  track fitting algorithm.
 This algorithm uses weighted sums of Gaussian
  functions to describe electron energy loss
  and thus allows for non-Gaussian corrections
  to the fitting of tracks.
 In the CMS detector, 
  bremsstrahlung from electrons results in the
  emmision of photons in an extended strip
  in the $\phi$ direction and electron
  superclusters (SCs) are made by 
  including the energy deposits from 
  these photons in the ECAL as part of the
  candidate electron object.

 Further requirements during the reconstruction of
  the electron improve the purity of selection.
 The SC and the GSF track are required to 
  be separated by no more than $\abs{\eta}<0.02$
  and $\abs{\phi}<0.15$ and the fraction of
  energy deposited in the HCAL directly behind
  the SC, and the SC is required to be no more
  than 15\%.
 

\section{Photon ID and Reconstruction}

 Photons are reconstructed using the same ECAL
  clustering algorithms as are used for electrons.
 This allows for
  the simultaneous reconstruction of
  photons that have and have not split to $e\overline{e}$
  pairs.
 The size of the SC is determined dynamically
  and the center is determined to be the barycenter
  of the distribution, with weights assigned
  using the logarithm of the fractional energy deposits
  of the ECAL crystals clustered in the SC.

 In an ideal tracker, photons would not interact 
  at all and objects that leave signarues
  similar to those of photons could be rejected
  through the rejection of tracks.
 However, some photons do convert to $e\overline{e}$
  pairs inside the tracker volume which leave tracks,
  so the rejection of tracks is not a perfect way 
  to distinguish between photons and electrons.

\section{Muon ID and Reconstruction}

 Muon identification is performed using
  two reconstruction and filtering methods to produce 
  `tracker muons' and `standalone muons' which are 
  combined to form  `global muons.'
 Tracker muons are identified starting with
  a track, $\pt>0.5$ \GeV and $p>2.5$ \GeV,
  which is then extrapolated to the muon system.
 If the distance between the the extrapolated
  track and the nearest hit in one of the muon 
  chambers is less than 3 cm, a tracker muon
  is identified.
 Tracker muons are also identified if the 
  pull between the extrapolated track and the
  matched station hit is less than four, where
  pull is defined as the distance between
  the track and the station hit divided by 
  the uncertainties on both measured quanties.
 Tracker muons are built from the inside of the
  detector towards the outside, and 
  standalone muons are built in the other direction.
 Only hits in the muon stations are used to 
  reconstruct standalone muons, with the 
  additional constraint that the path reconstructed
  from the hits points back toward the 
  interaction region.
 Thus, the tracker muon algorithm is well-suited
  for the identification of low-\pt muons by having
  low thresholds and requiring only one track
  and one station hit, while the standalone
  muon algorithm is aimed at high-\pt muons
  which have the energy to penetrate multiple layers
  of muon stations to form tracks which can be
  traced back to the interaction.
 Global muons are required to pass the criteria for both 
  standalone muons and tracker muons, and,
  starting with the standalone muons, 
  the global muon trajectory is refit using information from both 
  the muon stations and the tracker,
  yielding an improved energy resolution than either one.
 %For muons with $\pt<200$ \GeV, the resolution of the tracker




\section{Missing Transverse Energy}

\section{Jet ID and Secondary Vertices}

